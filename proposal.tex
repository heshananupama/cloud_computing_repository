\documentclass[11pt]{article}

\def\baselinestretch{1}
\usepackage{times}
\usepackage{titlesec}
\usepackage{fancyhdr}  % for displaying head/foot meta info 
\usepackage{pagecounting}
\usepackage{color}
\usepackage[yyyymmdd,hhmmss]{datetime}  % for using currenttime command 
\usepackage{hyperref}
\usepackage{lastpage}
\usepackage{lipsum}

\newcommand\bb[1]{\mbox{\em #1}}
\newcommand{\eat}[1]{}
\newcommand{\hsp}{\hspace*{\parindent}}
\definecolor{gray}{rgb}{0.4,0.4,0.4}


\titleformat{\section}{\vspace{- 0.5 \baselineskip}\normalfont\fontsize{12}{12}\bfseries}{\thesection}{1em}{}
\titleformat{\subsection}{\vspace{- 0.5 \baselineskip}\normalfont\fontsize{11}{11}\bfseries}{\thesubsection}{1em}{}

\begin{document}

\renewcommand{\headrulewidth}{0pt} 
\renewcommand{\footrulewidth}{0pt} 
\pagestyle{fancy}
\cfoot{}
\lhead{}
\rhead{}
\rfoot{\itshape\textcolor{gray}{Page \thepage\ of \pageref{LastPage}}}
\lfoot{\itshape\textcolor{gray}{CS525T Cloud Computing Paper Review}}

%%% Fill in the paper information %%% 
\begin{center}
{\LARGE \bf Project Proposal} \\
{\normalsize \emph{Heshan Perera, Michael Ludwig}}\\

\end{center}

\section{Problem Statement}

% Problem statement and motivation for the problem.
In our project, our goal is to compare realistic usage of virtual machines and containers in the cloud for end users. VMs are a necessity for cloud computing because providers need to share their resources between customers, which means customers are forced to use them. However, containers have proven to have several benefits in terms the application development lifecycle, making them appealing to cloud users.

In order to compare VMs and containers, we will 

- Fargate for containers and EC2 for VMs
- use cloudwatch for metrics
- measure performance
  - cpu
  - memory
  - disk
  - network
- Latency when scaling (due to startup time)
- Comparing single instance versus a cluster
  - finding a distributed benchmark application
- How easy it was deploy an application

\section{Related Work}

% Related work. For example, prior work that looked at similar problems or used similar techniques to what you are proposing. Focusing on the key differences between what you want to do and what have been done. This is an important aspect of the proposal that helps convince yourselves and me that spending 10 weeks on the project is worthwhile. 

- Containers and Virtual Machines at Scale: A Comparative Study
  - this paper used 5 different workloads; nothing ML related
  - tested VMs vs containers on bare metal and also containers inside VMs
    - concluded that containers have similar performance to VMs, except for disk IO, but VMs provide better isolation
  - Containers can't be used for multitenancy
  - Lightweight VMs reduce start up time (maybe try using Clear Linux)
  - VMs can do live migration, but cloud end-users likely don't need this feature when working with containers
  - container images have faster build time, smaller size, and versioning

https://ieeexplore.ieee.org/abstract/document/7095802

\section{Solution and Design}

- Fargate (on ECS) for containers and EC2 for VMs
- use cloudwatch for metrics
- Workloads
  - https://github.com/GoogleCloudPlatform/PerfKitBenchmarker
  - https://mlperf.org/training-overview (does both training and inference)
  - https://github.com/uillianluiz/RUBiS
  - https://github.com/brianfrankcooper/YCSB/


\section{Plan of Work}

% Plan of work. What key questions do you need to answer? How would you implement your solution? What type of resources do you need, e.g., machines, other equipment, access to datasets ? What are the expected timeline? 

2/10
Setup AWS accounts with student credits

2/17
2/24
Create container image and VM image for MLperf (Mike)
Create container image and VM image for Rubis (Heshan)

3/2
Run VM on AWS EC2 (figure out how to equally allocate resources)

3/9
Gather metrics from tool and AWS Cloudwatch

3/16
(stretch) Setup autoscaling groups for scaling benchmark
Work on mid term

3/23 (Mid term review)

3/30
4/6
Create container image and VM image for PerfKit (Mike)
Create container image and VM image for YCSB (Heshan)
(stretch) Create images that can serve requests for a fixed workload

4/13
Run VM on AWS EC2 (figure out how to equally allocate resources)
Gather metrics from tool and AWS Cloudwatch
(stretch) Write script that sends an increasing number of requests over a period of time

4/20
Work on presentation

4/27 (presentation)

1. Setup AWS accounts with student credits
2. Create container image for MLperf (Docker)
3. Create VM image for MLperf (Hashicorp Packer?)
4. Run container on AWS Fargate
5. Run VM on AWS EC2 (figure out how to equally allocate resources)
6. Repeat steps 2-5 for each single-instance benchmark

(if time permits)
7. Setup autoscaling groups for scaling benchmark
8. Create images that can serve requests for a fixed workload
9. Write script that sends an increasing number of requests over a period of time

10. Gather metrics from AWS Cloudwatch

\section{Evaluation Methodology}

- Compare results data from each benchmarking tool, if provided
- Compare resource usage data from Cloudwatch
- Metrics to compare:
  - How long the benchmark took to run
  - Utilization over time for each resource
  - Startup latency when scaling the amount of work

\section{Division of Work}

Split work by workloads, so we both get experience with both VMs and containers

Mike
  - https://github.com/GoogleCloudPlatform/PerfKitBenchmarker
  - https://mlperf.org/training-overview (does both training and inference)

Heshan
  - https://github.com/uillianluiz/RUBiS
  - https://github.com/brianfrankcooper/YCSB/

\end{document}
